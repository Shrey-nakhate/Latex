\documentclass[a4paper,11pt,twocolumn]{article}
\usepackage[a4paper,left=1.5cm,right=1cm,top=2cm,bottom=2cm]{geometry}
\usepackage{setspace}
\usepackage{gensymb}
\usepackage{caption}
\usepackage{graphicx}
\usepackage{tabularx}
\usepackage{lmodern}
\usepackage{watermark} 
\usepackage{lipsum}
\usepackage{xcolor}
\usepackage{listings}
\usepackage{graphicx}
\usepackage{enumitem}
\usepackage{mathtools}
\usepackage{titlesec}
\usepackage[utf8]{inputenc}
\usepackage{fontenc}
\usepackage{harvard}
\usepackage{amsfonts}
\graphicspath{{/storage/emulated/0/Download/FWC/Latex/figs}}
\usepackage[colorlinks,linkcolor={black},citecolor={blue!80!black},urlcolor={blue!80!black}]{hyperref}
\title{ GATE QUESTION Q.55(CS 2014 SET-C)}
\author{\textbf{\textit{\teflipflopxtbf{SHREY ANIL NAKHATE (1032222913)}}}}
\begin{document}

\date{}
\maketitle



\section{PROBLEM}
\textbf{(GATE CS-2014 Set C)}

\textbf{Q.5} Let ⊕ denote the Exclusive OR (XOR) operation. Let ‘1’ and ‘0’ denote the binary constants.
Consider the following Boolean expression for F over two variables P and Q: 
 
   F(P,Q) = ((1 − P) – (P + Q)) + ((P – Q) – (Q – 0))

The equivalent expression for F is 
\begin{enumerate}[label=(\Alph*)]
	\item $ P+Q$
        \item $ \overline{P+Q}$
        \item $ P\oplus Q$
	 \item $\overline{P \oplus Q}$
\end{enumerate}
\bigskip

\section{COMPONENTS}
	\begin{tabularx}{0.45\textwidth} {  
  | >{\centering\arraybackslash}X  
  | >{\centering\arraybackslash}X  
  | >{\centering\arraybackslash}X | } 
\hline 
\textbf{Component} &  \textbf{Value} & \textbf{Quantity}\\ 
\hline 
Arduino & UNO & 1 \\   
\hline 
Bread board & - & 1 \\ 
\hline 
Jumper wires & M-M & 20 \\ 
\hline 
Led light & - & 1\\ 
\hline 
Resistor & 150ohms & 1\\ 
\hline 
\end{tabularx}
\bigskip 

\section{INTRODUCTION}
\paragraph{}
	An "identity" is merely a relationship that is always true, regardless of the values that any variables involved might take on; similar to laws or properties. Many of these can be analogous to normal multiplication and addition, particularly when the symbols {0,1} are used for {FALSE, TRUE}. 
\bigskip 

\section{TRUTH TABLE}
The Truth Table for the answer $\overline{P \oplus Q}$  is given below
\bigskip
\bigskip
\begin{table}[ht!]
	\centering
\begin{tabular}{ |c |c |c |c |} 
\hline 
\newline 
	\textbf{x} & \textbf{y} & \textbf{Y1} & \textbf{Y2} \\
\hline  
	0 & 0 & 0 &1 \\   
	0 & 1 & 1 &0 \\ 
	1 & 0 & 1 &0 \\  
	1 & 1 & 0 &1\\  

\hline 
\end{tabular}
$Here Y1 = P \oplus Q$
$ ,Y2 = \overline{P \oplus Q}$
\caption{}
\end{table}
\bigskip
\bigskip

\section{ARDUINO CONNECTIONS}

1) The connections between Arduino and LED are as follows:
\begin{table}[ht!] 
    \centering 
    \begin{tabular}{|c|c|c|c|c|} 
    \hline 
       LED& $+ve$&$-ve$&$-ve$&$-$ \\ 
       \hline 
    ARDUINO& 2&4&Gnd&Vcc \\ 
    \hline 
    \end{tabular} 
    \caption{} 
\end{table} 
\bigskip
\bigskip
\bigskip
\section{CODE}
\paragraph{}
	The arduino code is given below.
\begin{center} 
\fbox{\parbox{8.5cm}{#include <Arduino.h>
\vskip
$int a,b,c,y;$
\vskip
$void setup()$
\vskip
${
 pinMode(2,OUTPUT);
 \vskip
 pinMode(4,INPUT);
 \vskip
 pinMode(5,INPUT);
 \vskip
 pinMode(6,INPUT);
 \vskip
 //pinMode(7,INPUT); 
 \vskip
 
 }$
void loop(){
\vskip
  a=digitalRead(4);
  \vskip
  b=digitalRead(5);
  \vskip
  c=digitalRead(6);
  \vskip
 // d=digitalRead(7);
 \vskip
    
 y= (a&&!b&&c)||(!a&&b&&c);
  \vskip
    
  digitalWrite(2,y);   
}}}
\end{center}


\end{document}
